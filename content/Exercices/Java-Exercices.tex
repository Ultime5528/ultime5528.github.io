%-----------------------------------------------------------
%-----------------------------------------------------------
%	Guide d'introduction à Java et WPILib
%	Par Étienne Beaulac
%	Ultime 5528
%	Mai 2017
%-----------------------------------------------------------
%-----------------------------------------------------------


\documentclass[12pt]{article}

%Code display, avant Babel !
%\usepackage{listings, minted}

\usepackage[francais]{babel}
\usepackage[utf8]{inputenc}
\usepackage[scaled]{helvet}
\renewcommand\familydefault{\sfdefault} 
%\usepackage{libertine}
%\usepackage{libertinust1math}
\usepackage[T1]{fontenc}
%\usepackage{lmodern}
%\usepackage{avant}

% Marges
\usepackage[margin=2cm]{geometry}

%Custom font sizes
\usepackage{anyfontsize}

%Tableau en français (et non table)
%\usepackage{caption}
%\captionsetup[table]{name=Tableau}

%packages graphiques, mathématiques
\usepackage{amsfonts, amsmath, amssymb}

%Afficher des images avec includegraphics
\usepackage{graphicx}

%Larger titles
\usepackage{titlesec}
\titleformat*{\section}{\LARGE\bfseries}

%Image avec caption
\newcommand{\image}[2]{%

}


%Utilisation de code R, après Babel pour éviter erreurs
%\lstloadlanguages{}


%Virgule pour nombres français
%\usepackage{icomma}

%Liens hypertextes
\usepackage{hyperref}
\hypersetup{
    colorlinks=true,
    linkcolor=blue,
    filecolor=magenta,      
    urlcolor=blue,
}

%Couleurs
\usepackage{xcolor}
\definecolor{ultRed}{RGB}{190,30,45}
\definecolor{ec-red}{RGB}{237,28,36}
\definecolor{ec-orange}{RGB}{255,127,39}
\definecolor{ec-yellow}{RGB}{255,242,0}
\definecolor{ec-green}{RGB}{34,177,76}
\definecolor{ec-purple}{RGB}{163,73,164}

%\usepackage{eso-pic}

%Nice code snippets
\usepackage[section]{minted}
\setminted[java]{%
	linenos,
	autogobble
}

%Nice frames for minted
\usepackage{tcolorbox}
\tcbuselibrary{minted, skins, xparse}

%Frame for commands
\newcommand{\commande}[1]{%
\tcbox[on line, size=fbox, colframe=black, boxrule=0.75pt, tcbox raise base]{#1} %boxsep=0pt, left=5pt, right=5pt, top=8pt, bottom=8pt
}

%Frame for minted listings
\newtcblisting[list inside=mybox, auto counter, number within=section]{MyTCB}[2][]{%
	colframe=ultRed,
	title={\textsc{Code \thetcbcounter} --- #2},
	sharp corners=south,
	boxsep=3mm,
	left=0.7cm,
	listing only,
	list text={#2},
	minted language=java,
	minted options={linenos, autogobble, baselinestretch=1, tabsize=4, breaklines=true}, #1}


% My own code snippet environment
\newenvironment{snippet}[1]
	{
	\begin{MyTCB}{#1}
	}
	{
	\end{MyTCB}
	}
	
%Commande pour images inline
\newcommand{\inlinepic}[1]{%
  \begingroup\normalfont
  \includegraphics[height=1em]{#1}%\fontcharht\font`\B
  \endgroup
}

%Listings caption
\renewcommand{\listingscaption}{Extrait de code}
\renewcommand{\listoflistingscaption}{Liste des extraits de code}

%Better aligned lists
\usepackage{scrextend}

%\usepackage{multirow}
%\usepackage{tikz}
%\usepackage{pgfplots}
%\pgfplotsset{width=7cm, compat=1.13}
%\usepgfplotslibrary{fillbetween}

%\setlength{\jot}{10pt} % Modifie l'espace entre les équations d'un bloc Align

%\showboxdepth=\maxdimen
%\showboxbreadth=\maxdimen


%Indentation nul au début de paragraphe
\setlength{\parindent}{0pt}

%Espace entre les paragraphes
\setlength{\parskip}{0.5\baselineskip}


%Espace entre les notes de bas de page
\setlength{\footnotesep}{0.95\baselineskip}

%----------------------------------------------------
%----------------------------------------------------
%----------------------------------------------------
% Début du document
%----------------------------------------------------
%----------------------------------------------------
%----------------------------------------------------

\begin{document}
%
%
%----------------------------------------------------
%----------------------------------------------------
% Page titre
%----------------------------------------------------
%----------------------------------------------------
%
%
\begin{titlepage}
	\vspace*{2.5cm}
	\hspace*{-2cm}\colorbox{ultRed}{%
	{\begin{minipage}{\paperwidth}	
		{\ \\[1cm] \hspace*{2cm} {\fontsize{40}{50}\selectfont Exercices} \\[10pt]
		\hspace*{2cm} {\Large Formation Java} \vspace*{1cm}}
	\end{minipage}}}\\[10pt]
	%
	\begin{minipage}{6cm}
		\raisebox{-0.1\height}{\parbox[b]{4cm}{\raggedleft {\large Étienne Beaulac\\[5pt] Ultime FRC 5528}\\[15pt] {\small Dernière modification\\ \today} }}%
		\hspace*{0.05\textwidth}%
		\raisebox{-0.5\height}{\rule{0.5pt}{6cm}}%
		\hspace*{0.32\textwidth}%
		\raisebox{-0.5\height}{\includegraphics[trim={2.6cm 0 2.6cm 0}, clip, height=7cm]{logo_ultime.png}}%
	\end{minipage}
\end{titlepage}
\pagebreak
%
\thispagestyle{empty}
\strut
\newpage
%
% Enlever les hyperliens bleus
{\hypersetup{hidelinks}
%
\clearpage
\pagenumbering{roman}
\setcounter{page}{1}

\tableofcontents

\newpage
%
%
\linespread{1.5}
\normalsize
\pagenumbering{arabic}
\setcounter{page}{1}
} % Fin hyperliens noirs


%----------------------------------------------------
% Ajouter fancyhf{}
%----------------------------------------------------

\begin{center}
\begin{tabular}{c c c}
\hline \\
\quad & Dans tous vos programmes, vous devez utiliser & \quad \\
& des identificateurs significatifs (noms de variables, etc.)\\
& et commenter votre code de manière appropriée.\\
\ \\
\hline
\end{tabular}
\end{center}




%---------------------------------------------------
%
% Console et variables
%
%---------------------------------------------------

\section{Interactions avec la console et variables}
\vspace*{-\baselineskip}
Séance du 10 mai 2017. Références du manuel : p. 1 à 38

%
%
%

\subsection{Présentation}

Votre programme doit demander à l'utilisateur son \textbf{nom}, son \textbf{âge} et son \textbf{salaire}. Par la suite, il doit le réafficher dans la console.

Exemple de sortie console :

{\footnotesize \fontfamily{pcr}\selectfont
\begin{tabular}{|p{0.98\textwidth}|}
\hline 
Saisissez votre nom : \textcolor{blue}{Jonathan}\\
Saisissez votre âge : \textcolor{blue}{24}\\
Saisissez votre salaire : \textcolor{blue}{17.45}\\[\baselineskip]

Bonjour Jonathan! Vous avez 24 ans et votre salaire est de 17.45 \$ par heure.\\
\hline
\end{tabular}
}


%
%
%

\subsection{Aire d'un rectangle}

Votre programme doit demander à l'utilisateur de saisir une largeur et une hauteur, puis retourner l'aire du rectangle correspondant.

%
%
%
%

\subsection{Moyenne}

Votre programme doit permettre à l’utilisateur de saisir cinq nombres, puis calculer et afficher la moyenne de ces nombres.

\textit{Défi : Soyez astucieux et tentez d'utiliser une seule variable!}




%---------------------------------------------------
%
% Structures conditionnelles
%
%---------------------------------------------------

\section{Structures conditionnelles}
\vspace*{-\baselineskip}
Séance du ... mai 2017. Références du manuel : p. 39 à 43

\subsection{Résultats à l'examen}

À partir d'une note sur 100 saisie par l'utilisateur, affichez un message correspondant :

\begin{itemize}
	\item[$\bullet$] 100\% : Affichez qu'il s'agit d'une note parfaite.
	\item[$\bullet$] Plus de 60\% (sauf 100\%) : Affichez que l'utilisateur a réussi l'examen.
	\item[$\bullet$] Moins de 60\% : Affichez qu'il s'agit d'un échec et indiquez le pourcentage qu'il manquait à l'utilisateur pour avoir 60\%.
	\item[$\bullet$] Note qui n'est pas comprise et 0 et 100 : Affichez un message d'erreur.
\end{itemize}

% 
%
%

\subsection{Compagnie de téléphone}

À partir du nombre de minutes utilisées saisi par l’utilisateur, calculez et affichez le prix de la facture de téléphone selon les modalités ci-dessous. \textbf{Vous devez utiliser des constantes lorsque possible.}

\begin{itemize}
	\item[$\bullet$] La compagnie facture un montant initial de 10\$ par mois.
	\item[$\bullet$] Les 30 premières minutes sont facturées à un prix de 0,20\$ par minute.
	\item[$\bullet$] Les minutes suivantes sont facturées à un prix de 0,10\$ par minute.
\end{itemize}

Par exemple, une utilisation de 44 minutes serait facturée 17,40\$.

%
%
%
%

\subsection{Conversion Celsius - Fahrenheit}

Votre programme doit demander à l'utilisateur s'il souhaite convertir une température des Celsius vers les Fahrenheit ou des Fahrenheit vers les Celsius. Il doit ensuite pouvoir saisir sa température et obtenir le résultat.
%
\begin{align*}
	\text{Celsius vers Fahrenheit :} \quad F &= \frac{9}{5} \times C + 32\\[5pt]
	\text{Fahrenheit vers Celsius :} \quad C &= \frac{5}{9} \times (F - 32)
\end{align*}


\end{document}






\section{Conversion d'années en mois et jours}

Écrire un programme qui, à partir d'une variable \commande{age}, affiche le nombre correspondants de mois et de jours. Ne tenez pas compte des années bissextiles. Vous devez :

\begin{itemize}
\item Utiliser deux constantes pour la conversion en mois et en jours.
\item Utiliser trois variables différentes pour contenir les trois durées.
\end{itemize}

\begin{MyTCB}{ConversionAge.java}
/**
 * Conversion d'années en mois et en jours
 * @author Etienne
 *
 */
public class ConversionAge {

	public static void main(String[] args) {
		
		int age = 10;
		int mois, jours;
		
		final int MOIS_DANS_ANNEE = 12;
		final int JOURS_DANS_MOIS = 365;
		
		//Conversion
		mois = age * MOIS_DANS_ANNEE;
		jours = mois * JOURS_DANS_MOIS;
		
		//Affichage
		System.out.println("Bonjour! J'ai " + age + " ans,");
		System.out.println("soit " + mois + " mois et " + jours + " jours.");

	}

}
\end{MyTCB}





\begin{tcolorbox}[adjusted title=Titre 1]
Allo !
\end{tcolorbox}
%
\begin{tcblisting}{}
	Test!
\end{tcblisting}





%Enlever les trop grands espaces entre les titres
\usepackage{titlesec}

\titleformat{\chapter}[display]   
{\normalfont\huge\bfseries}{\chaptertitlename\ \thechapter}{20pt}{\Huge}   
\titlespacing*{\chapter}{0pt}{\parskip}{-\parskip}
\titlespacing{\section}{0pt}{\parskip}{-\parskip}
\titlespacing{\subsection}{0pt}{\parskip}{-\parskip}
\titlespacing{\subsubsection}{0pt}{\parskip}{-\parskip}





