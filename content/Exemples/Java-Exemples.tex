%-----------------------------------------------------------
%
%	Exemples Java
%	Par Étienne Beaulac
%	Ultime 5528
%	Mai 2017
%
%-----------------------------------------------------------


\documentclass[12pt]{article}

%Code display, avant Babel !
%\usepackage{listings, minted}

\usepackage[francais]{babel}
\usepackage[utf8]{inputenc}
\usepackage[scaled]{helvet}
\renewcommand\familydefault{\sfdefault} 
%\usepackage{libertine}
%\usepackage{libertinust1math}
\usepackage[T1]{fontenc}
%\usepackage{lmodern}
%\usepackage{avant}

% Marges
\usepackage[margin=2cm]{geometry}

%Custom font sizes
\usepackage{anyfontsize}

%Tableau en français (et non table)
%\usepackage{caption}
%\captionsetup[table]{name=Tableau}

%packages graphiques, mathématiques
%\usepackage{amsfonts, amsmath, amssymb}

%Afficher des images avec includegraphics
\usepackage{graphicx}

%Image avec caption
\newcommand{\image}[2]{%

}


%Utilisation de code R, après Babel pour éviter erreurs
%\lstloadlanguages{}


%Virgule pour nombres français
%\usepackage{icomma}

%Liens hypertextes
\usepackage{hyperref}
\hypersetup{
    colorlinks=true,
    linkcolor=blue,
    filecolor=magenta,      
    urlcolor=blue,
}

%Couleurs
\usepackage{xcolor}
\definecolor{ultRed}{RGB}{190,30,45}
\definecolor{ec-red}{RGB}{237,28,36}
\definecolor{ec-orange}{RGB}{255,127,39}
\definecolor{ec-yellow}{RGB}{255,242,0}
\definecolor{ec-green}{RGB}{34,177,76}
\definecolor{ec-purple}{RGB}{163,73,164}

%\usepackage{eso-pic}

%Nice code snippets
\usepackage[section]{minted}
\setminted[java]{%
	linenos,
	autogobble
}

%Nice frames for minted
\usepackage{tcolorbox}
\tcbuselibrary{minted, skins, xparse, breakable}

%Frame for commands
\newcommand{\commande}[1]{%
\tcbox[on line, size=fbox, colframe=black, boxrule=0.75pt, tcbox raise base]{#1} %boxsep=0pt, left=5pt, right=5pt, top=8pt, bottom=8pt
}

%Frame for minted listings
\newtcblisting[list inside=mybox, auto counter, number within=section]{MyTCB}[2][]{%
	colframe=ultRed,
	title={\textsc{Code \thetcbcounter} --- #2},
	sharp corners=south,
	boxsep=3mm,
	left=0.7cm,
	listing only,
	list text={#2},
	breakable, enhanced,
	minted language=java,
	minted options={linenos, autogobble, baselinestretch=1, tabsize=4, breaklines=true}, #1}


% My own code snippet environment
\newenvironment{snippet}[1]
	{
	\begin{MyTCB}{#1}
	}
	{
	\end{MyTCB}
	}
	
%Commande pour images inline
\newcommand{\inlinepic}[1]{%
  \begingroup\normalfont
  \includegraphics[height=1em]{#1}%\fontcharht\font`\B
  \endgroup
}

%Listings caption
\renewcommand{\listingscaption}{Extrait de code}
\renewcommand{\listoflistingscaption}{Liste des extraits de code}

%Better aligned lists
\usepackage{scrextend}

%\usepackage{multirow}
%\usepackage{tikz}
%\usepackage{pgfplots}
%\pgfplotsset{width=7cm, compat=1.13}
%\usepgfplotslibrary{fillbetween}

%\setlength{\jot}{10pt} % Modifie l'espace entre les équations d'un bloc Align

%\showboxdepth=\maxdimen
%\showboxbreadth=\maxdimen


%Indentation nul au début de paragraphe
\setlength{\parindent}{0pt}

%Espace entre les paragraphes
\setlength{\parskip}{0.5\baselineskip}


%Espace entre les notes de bas de page
\setlength{\footnotesep}{0.95\baselineskip}

%----------------------------------------------------
%----------------------------------------------------
%----------------------------------------------------
% Début du document
%----------------------------------------------------
%----------------------------------------------------
%----------------------------------------------------

\begin{document}
%
%
%----------------------------------------------------
%----------------------------------------------------
% Page titre
%----------------------------------------------------
%----------------------------------------------------
%
%
\begin{titlepage}
	\vspace*{2.5cm}
	\hspace*{-2cm}\colorbox{ultRed}{%
	{\begin{minipage}{\paperwidth}	
		{\ \\[1cm] \hspace*{2cm} {\fontsize{40}{50}\selectfont Exemples} \\[10pt]
		\hspace*{2cm} {\Large Formation Java et WPILib} \vspace*{1cm}}
	\end{minipage}}}\\[10pt]
	%
	\begin{minipage}{6cm}
		\raisebox{-0.1\height}{\parbox[b]{4cm}{\raggedleft {\large Étienne Beaulac\\[5pt] Ultime FRC 5528}\\[15pt] {\small Dernière modification\\ \today} }}%
		\hspace*{0.05\textwidth}%
		\raisebox{-0.5\height}{\rule{0.5pt}{6cm}}%
		\hspace*{0.32\textwidth}%
		\raisebox{-0.5\height}{\includegraphics[trim={2.6cm 0 2.6cm 0}, clip, height=7cm]{logo_ultime.png}}%
	\end{minipage}
\end{titlepage}
\pagebreak
%
\thispagestyle{empty}
\strut
\newpage
%
% Enlever les hyperliens bleus
{\hypersetup{hidelinks}
%
\clearpage
\pagenumbering{roman}
\setcounter{page}{1}

\tableofcontents

\newpage
%
%
\linespread{1.5}
\normalsize
\pagenumbering{arabic}
\setcounter{page}{1}
} % Fin hyperliens noirs


%----------------------------------------------------
% Ajouter fancyhf{}
%----------------------------------------------------

%----------------------------------------------------
%----------------------------------------------------
% Première séance
%----------------------------------------------------
%----------------------------------------------------
%
%
\section{Interactions avec la console et variables}
\vspace*{-\baselineskip}
Séance du 10 mai 2017. Références manuel : p. 1 à 38

\subsection{Affichage dans la console}

Écrire un programme qui affiche un message dans la console.

\begin{MyTCB}{MonPremierProgramme.java}
/**
 * Affiche un message dans la console.
 * @author Etienne
 *
 */ 
public class MonPremierProgramme {
	
	public static void main(String[] args) {
	
		//Affichage du message
		System.out.println("Hello world!");
		
	}

}
\end{MyTCB}
\newpage
%
%
%
%
%
\subsection{Affichage du nom de l'utilisateur}

Écrire un programme qui demande à l'utilisateur son nom, puis qui l'affiche.

\begin{MyTCB}{AffichageNom.java}
import java.util.Scanner;

/**
 * Demande le nom de l'utilisateur, puis l'affiche.
 * 
 * @author Etienne
 */
public class AffichageNom {

	public static void main(String[] args) {
		
		String nom;
		Scanner scanner = new Scanner(System.in);
		
		//Demander le nom
		System.out.print("Saisissez votre nom : ");
		nom = scanner.nextLine();
		
		//Affichage
		System.out.println("Votre nom est " + nom + "!");

	}

}
\end{MyTCB}
\newpage
%
%
%
%
%
\subsection{Nombre d'années avant la majorité}

Écrire un programme qui demande à l'utilisateur son âge et qui renvoie le nombre d'années avant qu'il soit majeur.

\begin{MyTCB}{Majorite.java}
import java.util.Scanner;

/**
 * Demande l'âge de l'utilisateur et affiche le nombre
 * d'années avant qu'il soit majeur.
 * 
 * @author Etienne
 */
public class Majorite {

	public static void main(String[] args) {
		
		int age;
		Scanner scanner = new Scanner(System.in);
		
		//Demander l'âge
		System.out.print("Saisissez votre âge : ");
		age = scanner.nextInt();
		
		//Âge avant majorité
		System.out.println("Vous serez majeur dans " + (18 - age) + " ans.");

	}

}
\end{MyTCB}
%
%
%
%
%----------------------------------------------------
%----------------------------------------------------
% Deuxième séance
%----------------------------------------------------
%----------------------------------------------------
%
\section{Constantes, classe \emph{Math} et structures conditionnelles}
\vspace*{-\baselineskip}
Séance du 15 mai 2017. Références du manuel : p. 39 à 43


\subsection{Valeur absolue et exposant}

Utiliser quelques méthodes de la class \emph{Math}.

\begin{MyTCB}{TestMath.java}
import java.util.Scanner;

/**
 * Affiche la valeur absolue et le cube d'un nombre.
 * 
 * @author Etienne
 *
 */
public class TestMath {

	public static void main(String[] args) {
		
		double nombre;
		Scanner scanner = new Scanner(System.in);
		
		//Obtention du nombre
		System.out.print("Saisissez un nombre : ");
		nombre = scanner.nextDouble();
		
		//Calculs
		System.out.println("\nLa valeur absolue du nombre est : " + Math.abs(nombre));
		System.out.println("Le cube du nombre est : " + Math.pow(nombre, 3));

	}

}
\end{MyTCB}
%
%
%

\subsection{Messages selon l'âge}

Écrire un programme qui demande l'âge de l'utilisateur et qui affiche un message le nombre obtenu. 

\begin{itemize}
	\item[$\bullet$] Entre 0 et 4 ans, il n'est pas encore à l'école.
	\item[$\bullet$] Entre 5 et 11 ans, il est au primaire.
	\item[$\bullet$] Entre 12 et 17 ans, il est au secondaire.
	\item[$\bullet$] À partir de 18 ans, il est majeur.
	\item[$\bullet$] Tout autre âge affiche un message d'erreur.
\end{itemize}

Vous devez utiliser trois constantes : \verb|AGE_PRIMAIRE = 5|, \verb|AGE_SECONDAIRE = 12| et \verb|AGE_MAJORITE = 18|.


\begin{MyTCB}{MessageAge.java}
import java.util.Scanner;
/**
 * Affiche un message selon l'âge de l'utilisateur.
 * 
 * @author Etienne
 */
public class MessageAge {

	public static void main(String[] args) {
		
		int age;
		final int AGE_PRIMAIRE = 5, AGE_SECONDAIRE = 12, AGE_MAJORITE = 18;
		Scanner scanner = new Scanner(System.in);
		
		//Obtention de l'âge
		System.out.print("Saisissez votre âge : ");
		age = scanner.nextInt();
		
		//Message selon l'âge
		if(age < 0)
			System.out.println("Âge invalide!");
		
		else if(age < AGE_PRIMAIRE)
			System.out.println("Pas encore à l'école!");
		
		else if(age < AGE_SECONDAIRE)
			System.out.println("Au primaire!");
		
		else if(age < AGE_MAJORITE)
			System.out.println("Au secondaire!");
		
		else
			System.out.println("Vous êtes majeur!");
		
	}

}
\end{MyTCB}

%
%
%
%

\subsection{Validation d'une année de naissance}

Votre programme doit demander une année de naissance à l'utilisateur, puis la valider. On considère que l'année minimale est 1900 et que quelqu'un ne peut pas être né plus tard que cette année. Utilisez des constantes lorsque possible.

\begin{MyTCB}{ValidationNaissance.java}
import java.util.Calendar;
import java.util.Scanner;
/**
 * Validation d'une année de naissance.
 * 
 * @author Etienne
 */
public class ValidationNaissance {

	public static void main(String[] args) {
		
		int annee;
		final int ANNEE_MINIMALE = 1900;
		final int ANNEE_COURANTE = Calendar.getInstance().get(Calendar.YEAR); //2017
		Scanner scanner = new Scanner(System.in);
		
		
		//Obtention de l'année
		System.out.print("Saisissez une année de naissance : ");
		annee = scanner.nextInt();
		
		
		//Validation
		if(annee >= ANNEE_MINIMALE && annee <= ANNEE_COURANTE)
			System.out.println("Année valide.");
		
		else {
			System.out.println("L'année " + annee + " est invalide.");
			System.out.println("Vous devez recommencer!");
		}

	}

}
\end{MyTCB}



%---------------------------------------------------
% Séance #3 - Méthodes (fonctions
%---------------------------------------------------

\section{Les méthodes (fonctions)}
\vspace*{-\baselineskip}
Séance du 24 mai 2017. Références du manuel : p. 69 à 77

\subsection{Dire bonjour}

Écrire et utiliser une méthode qui affiche un message de bienvenue dans la console.

\begin{MyTCB}{Bonjour.java}
/**
 * Dire bonjour dans la console
 *
 * @author Etienne
 */
public class Bonjour {

	public static void main(String[] args) {
		
		direBonjour();
		
	}
	
	
	/**
	 * Affiche un message de bienvenue dans la console.
	 */
	public static void direBonjour() {
		
		System.out.println("Bonjour!");
		
	}
	
}

\end{MyTCB}

%
%
%
%
%

\subsection{Ajouter deux à un entier}

Écrire et utiliser une méthode qui reçoit un entier en paramètre et qui lui ajoute deux. La méthode doit ensuite renvoyer ce nombre.

\begin{MyTCB}{AjouteDeux.java}
import java.util.Scanner;

/**
 * Utilisation d'une méthode qui ajoute 2
 * à un entier.
 *
 * @author Etienne
 */
public class AjouteDeux {

	public static void main(String[] args) {
		
		Scanner scanner = new Scanner(System.in);
		int nbre;
		
		//Obtention d'une valeur
		System.out.print("Saisissez un entier : ");
		nbre = scanner.nextInt();
		
		//Calcul et affichage
		nbre = ajouteDeux(nbre);
		System.out.println("Nouvelle valeur : " + nbre);
		
	}
	
	
	/**
	 * Additionne 2 à un entier.
	 * @param nombre le nombre à additionner
	 * @return le nombre augmenté de 2
	 */
	public static int ajouteDeux(int nombre) {
		
		int valeur = nombre + 2;
		
		return valeur;
		
		//return nombre + 2;
	}
	
}
\end{MyTCB}

%
%
%
%
%
%

\subsection{Méthode \emph{demanderNom}}

Écrire et utiliser une méthode qui demande un nom et qui retourne le nom saisi.\\
\emph{Défi : vérifier si le nom est vide (ne contient que des espaces).}

\begin{MyTCB}{DemanderNom.java}
import java.util.Scanner;

/**
 * Demande 3 noms à l'utilisateur et les affiche.
 * 
 * @author Etienne
 *
 */
public class DemanderNom {

	
	public static void main(String[] args) {

		String nom1, nom2, nom3;
		
		nom1 = demanderNom();
		nom2 = demanderNom();
		nom3 = demanderNom();
		
		System.out.println(nom1 + ", " + nom2 + " et " + nom3 + "!");		

	}
	
	
	/**
	 * Demande un nom à l'utilisateur.
	 * Affiche un message d'erreur si le nom ne contient que des espaces.
	 * 
	 * @return Le nom saisi.
	 */
	public static String demanderNom() {
		
		Scanner sc = new Scanner(System.in);
		
		System.out.print("Écrivez un nom : ");
		String nom = sc.nextLine();
		
		if(nom.trim().isEmpty())
			System.out.println("Ce nom est vide!");
		
		return nom;
		
	}

}

\end{MyTCB}

\subsection{Nombre de secondes}

Écrire et utiliser une méthode qui reçoit en paramètres un nombre d'heures, de minutes et de secondes et qui retourne le nombre total de secondes.

\begin{MyTCB}{DureeSecondes.java}
import java.util.Scanner;

/**
 * Utilisation de la méthode nombreDeSecondes, qui
 * transforme une durée en nombre de secondes.
 *
 * @author Etienne
 *
 */
public class DureeSecondes {

	
	public static void main(String[] args) {

		
		//Première durée
		System.out.println("Il y a " + nombreDeSecondes(24, 0, 0) + " secondes"
				+ " dans une journée.");
		
		
		//Deuxième durée
		int h = 12, m = 32, s = 17, duree;
		duree = nombreDeSecondes(h, m, s);
		
		System.out.println("Il y a " + duree + " secondes dans " 
				+ h + " heures, " + m + " minutes et " + s + " secondes.");

		
		//Troisième durée
		Scanner sc = new Scanner(System.in);
		
		System.out.print("Heures : ");
		h = sc.nextInt();
		
		System.out.print("Minutes : ");
		m = sc.nextInt();
		
		System.out.print("Secondes : ");
		s = sc.nextInt();
		
		System.out.println("Durée totale : " + nombreDeSecondes(h, m, s) + " secondes.");
		
	}
	
	
	/**
	 * Calcule le nombre de secondes d'une durée déterminée.
	 * 
	 * @param heures Le nombre d'heures.
	 * @param minutes Le nombre de minutes.
	 * @param secondes Le nombre de secondes.
	 * 
	 * @return La durée totale.
	 */
	public static int nombreDeSecondes(int heures, int minutes, int secondes) {
		
		int temps = 0;
		
		temps += heures * 60 * 60;
		temps += minutes * 60;
		temps += secondes;
		
		return temps;
		
	}

}
\end{MyTCB}
\subsection{Méthode \emph{estMajeur}}

Écrire et utiliser une méthode qui reçoit en paramètre un âge et qui renvoit le booléen VRAI si la personne est majeure, FAUX dans le cas contraire. On considère que l'âge de majorité est 18 ans.

\begin{MyTCB}{Majorite.java}
import java.util.Scanner;

/**
 * Détermine si l'utilisateur est majeur.
 * 
 * @author Etienne
 */
public class Majorite {

	public static void main(String[] args) {
		
		Scanner scanner = new Scanner(System.in);
		int ageUtilisateur;
		
		//Obtention d'une valeur
		System.out.print("Saisissez votre âge : ");
		ageUtilisateur = scanner.nextInt();
		
		//Vérification majorité
		if(estMajeur(ageUtilisateur)) {
			
			System.out.println("Wow!");
			System.out.println("Vous êtes majeur!");
			
		}
		else {
			System.out.println("Vous n'êtes pas encore majeur.");
		}
		
	}
	
	
	/**
	 * Détermine si une personne est majeure selon son âge.
	 * @param age L'âge à vérifier
	 * @return VRAI si la personne est majeure, FAUX dans le cas contraire
	 */
	public static boolean estMajeur(int age) {
		
		boolean majeur;
		
		if(age >= 18)
			majeur = true;
		else
			majeur = false;
		
		return majeur; //return (age >= 18);
	}
	
}
\end{MyTCB}

%
%
%
%
%

\subsection{Méthode \emph{estMajeur} surchargée}

Ajouter une méthode surchargée à votre programme précédent. La nouvelle méthode \verb|estMajeur| doit avoir deux paramètres : l'âge de l'utilisateur et l'âge nécessaire pour être majeur.

\begin{MyTCB}{Majorite2.java}
import java.util.Scanner;

/**
 * Détermine si l'utilisateur est majeur au Canada et aux États-Unis.
 * 
 * @author Etienne
 */
public class Majorite2 {

	public static void main(String[] args) {
		
		Scanner scanner = new Scanner(System.in);
		int ageUtilisateur;
		
		//Obtention de l'âge
		System.out.print("Saisissez votre âge : ");
		ageUtilisateur = scanner.nextInt();
		
		//Vérification majorité par défaut
		if(estMajeur(ageUtilisateur)) {
			
			System.out.println("Wow!");
			System.out.println("Vous êtes majeur au Canada!");
			
		}
		else {
			System.out.println("Vous n'êtes pas encore majeur au Canada.");
		}
		
		
		//Vérification majorité américaine
		if(estMajeur(ageUtilisateur, 21))
			System.out.println("Vous êtes majeur aux États-Unis.");
		
		else
			System.out.println("Vous n'êtes pas encore majeur aux États-Unis.");
		
	}
	
	
	/**
	 * Détermine si une personne est majeure selon son âge (par défaut à 18 ans). 
	 * @param age L'âge à vérifier.
	 * @return VRAI si la personne est majeure, FAUX dans le cas contraire.
	 */
	public static boolean estMajeur(int age) {
		
		return estMajeur(age, 18);
		
	}
	
	
	/**
	 * Détermine si une personne est majeure selon son âge.
	 * @param age L'âge à vérifier.
	 * @param ageMajorite L'âge pour être majeur.
	 * @return VRAI si la personne est majeure, FAUX dans le cas contraire.
	 */
	public static boolean estMajeur(int age, int ageMajorite) {
		
		boolean majeur;
		
		if(age >= ageMajorite)
			majeur = true;
		else
			majeur = false;
		
		return majeur; //return (age >= ageMajorite);
		
	}
	
}
\end{MyTCB}

\end{document}






% Faire demander l'âge

\section{Conversion d'années en mois et jours}

Écrire un programme qui, à partir d'une variable \commande{age}, affiche le nombre correspondants de mois et de jours. Ne tenez pas compte des années bissextiles. Vous devez :

\begin{itemize}
\item Utiliser deux constantes pour la conversion en mois et en jours.
\item Utiliser trois variables différentes pour contenir les trois durées.
\end{itemize}

\begin{MyTCB}{ConversionAge.java}
/**
 * Conversion d'années en mois et en jours
 * @author Etienne
 *
 */
public class ConversionAge {

	public static void main(String[] args) {
		
		int age = 10;
		int mois, jours;
		
		final int MOIS_DANS_ANNEE = 12;
		final int JOURS_DANS_MOIS = 365;
		
		//Conversion
		mois = age * MOIS_DANS_ANNEE;
		jours = mois * JOURS_DANS_MOIS;
		
		//Affichage
		System.out.println("Bonjour! J'ai " + age + " ans,");
		System.out.println("soit " + mois + " mois et " + jours + " jours.");

	}

}
\end{MyTCB}





\begin{tcolorbox}[adjusted title=Titre 1]
Allo !
\end{tcolorbox}
%
\begin{tcblisting}{}
	Test!
\end{tcblisting}





%Enlever les trop grands espaces entre les titres
\usepackage{titlesec}

\titleformat{\chapter}[display]   
{\normalfont\huge\bfseries}{\chaptertitlename\ \thechapter}{20pt}{\Huge}   
\titlespacing*{\chapter}{0pt}{\parskip}{-\parskip}
\titlespacing{\section}{0pt}{\parskip}{-\parskip}
\titlespacing{\subsection}{0pt}{\parskip}{-\parskip}
\titlespacing{\subsubsection}{0pt}{\parskip}{-\parskip}





